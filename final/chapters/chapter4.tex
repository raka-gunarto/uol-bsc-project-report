\chapter{Discussion}
\label{chapter4}

\section{Conclusions}
The objective of the project, implementing graphics for the Xv6 operating system,
has been achieved to a satisfactory level. The solution implementation is 
able to communicate with a VGA compatible PCI hardware, set it up properly for graphics,
and expose the display in the form of a grid of windows to user programs. Furthermore, a 
graphics library that further abstracts the intricacies of allocating a window
and drawing graphics was developed, creating a simple interface for user program
development.
\section{Ideas for future work}
Although satisfactory, some aspects of the project can be developed further.
\subsection{Portability}
While the code is mostly portable, some parts were hardcoded to stick to the project
timeframe and to stay within the project scope. 

Addresses for communication with PCI devices were hardcoded to only support the
QEMU 'virt' emulated board. These addresses may be different on other systems,
and the proper implementation should be to read the device tree blob placed
in memory to retrieve the addresses.

The VGA framebuffer is exposed at a constant location in the PCI MMIO address space.
This may cause conflicts if there are other PCI devices also using the PCI MMIO.
The PCI code can be improved to keep track of used address spaces and allocate the
base addresses dynamically.

The legacy PCI access method is also being used, which may limit discovery of PCI
hardware.
\subsection{Further VGA support}
Currently only a limited subset of VGA features are implemented. The driver only
supports a single mode, mode 13h, which is a 320x200 resolution 256 colour mode.
The implementation could be further improved to implement other modes, or even
the extended standards such as SVGA for larger resolutions and a larger colour
space.

The VGA driver is also very primitive, it only supports copying memory into the framebuffer.
`Bit blitting' operations could be implemented in the future, where the source data 
and the framebuffer is combined through bitwise operations. This could greatly optimise
draw operations for non rectangular shapes, as the buffer could still be rectangular
but pixels can be left unaffected.

\subsection{Human interfacing}
Although slightly out of scope, more human interfacing options could be provided. This
could be done by adding USB support for mice, keyboard, gamepads, etc. A mouse cursor
could also be implemented in the future. 

\subsection{Advanced graphics library}
The current implementation for the graphics library is not ideal for high framerate
applications. Double buffering could be implemented to only update the real window 
framebuffer when it is ready to be displayed, reducing flickering or other
graphical artifacts and anomalies.

More utilities could also be included, such as drawing triangle primitives and 
maybe 2D space transformations. This was planned to be included in the project,
but unfortunately there were some issues in QEMU with floating point operations,
resulting in the functionality being removed.


