\chapter{Methods}
\label{chapter2}

\section{Development plan}
\subsection{Model}
The waterfall model will be used following the linear nature of the project. 
Features in the planned software are linearly dependent, and following this model
will allow consistent progress to reach the planned objectives and deliverables.

\subsection{Version control}
As Xv6 is already in a git repository hosted on GitHub \cite{xv6:code}, 
we can simply fork the repository for our modifications. Using git will allow
the progress of the software implementation to be easily documented.

\section{Analysis and design}

\subsection{Running Xv6}
There are two options of running Xv6. This subsection will explore the pros
and cons of both options in this project's use case.
The usual choice is to use QEMU, an emulator that can run systems built for the RISC-V architecture. 

\subsubsection*{Pros}
\begin{itemize}
    \item Xv6 built with QEMU in mind
    \item Extensive debugging capabilities
    \item Easy graphics card and display emulation
    \item Easy to install from the Ubuntu software repositories
\end{itemize}
\subsubsection*{Cons}
\begin{itemize}
    \item Not very performant
\end{itemize}

The alternative option is to run it on bare metal. As there aren't many 
RISC-V processors, the options are limited, with one of the only options being
the `HiFive Unmatched' board by SiFive \cite{hifiveunmatched}.

\subsubsection*{Pros}
\begin{itemize}
    \item Dedicated RISC-V processor, very performant
    \item Ability to observe the solution on real hardware
\end{itemize}
\subsubsection*{Cons}
\begin{itemize}
    \item Expensive, costs USD 665
    \item Would require additional hardware for this project, adding to the already expensive cost of the board
    \begin{itemize}
        \item VGA compatible graphics card
        \item JTAG interface hardware for debugging
    \end{itemize}
    \item Modifications in Xv6 required as some memory addresses are hardcoded for the emulated QEMU `virt' board
\end{itemize}


Given all this, running Xv6 in QEMU is the best option as it will suffice for 
development and testing of a graphics implementation.

\subsection{QEMU VGA emulation}
There are many emulation options for graphics provided by QEMU \cite{qemucards}. 
We will choose the most basic VGA card that we can access through PCI communication.
This card provides full VGA compatibility which can be configured through its data
ports (using the memory mapped PCI port IO addresses), and a linear framebuffer exposed
as memory mapped IO which can be mapped anywhere in the PCI MMIO region of the virtual
address space.

\subsection{Solution diagram}
\begin{figure}[H]
    \centering
    \includegraphics[width=16cm]{diagram.drawio.png}
    \caption{Solution design diagram}
    \label{figure:3}
\end{figure}

The design diagram in figure \ref{figure:3} illustrates how the different parts
of the solution will communicate and interact with each other. In short, the PCI
discovery code will find the VGA card and configure the base addresses, then signal
the driver when the VGA device is ready. 

The VGA driver will provide an interface to the framebuffer and some modesetting code 
through the port IO and memory IO which
have address mappings in the virtual address space. The window manager will then use
the framebuffer exposed by the driver to process user code that draws to the screen,
and register UART interrupts as input events to be passed back to user code.

\section{Implementation}

\subsection{Xv6 build modifications}
A Makefile is provided with the Xv6 project that contains tasks that builds the
kernel, user programs, and the filesystem. It also has tasks that runs the compiled
kernel and filesystem image on QEMU.

\subsubsection{QEMU options}
There is a variable called \mintinline{c}{QEMUOPTS} in the Makefile that contains
command line options forwarded to QEMU. We need to add the following line
\begin{minted}{Makefile}
    QEMUOPTS += -device VGA -vnc :0
\end{minted}
to enable the VGA device emulation and serve the display output via VNC. We will also
modify the C compiler flags to include \mintinline{bash}{-O0} to disable all optimisations
during development which could otherwise interfere with debugging.

When creating new files to be compiled into the kernel 
(Chapters \ref{chapter2:impl:pci},\ref{chapter2:impl:driver},\ref{chapter2:impl:winman}),
the appropriate compile targets must be appended to the \mintinline{c}{OBJS} variable. Similarly
for user programs, shared code must be appended to \mintinline{c}{ULIB}, and the programs appended
to \mintinline{c}{UPROGS}.

\subsection{Xv6 memory mapping}
\label{chapter2:impl:memory}
As mentioned earlier in Chapter \ref{chapter1:research:hardware}, the memory addresses for
PCI communication will need to be configured in Xv6. Using the mappings found in the
QEMU source tree in Listing \ref{listing:1}, we can add the following lines into the kernel
pagetable in kernel/vm.c (see Appendix \ref{appendix:c:2} for address definitions).
\begin{minted}[breaklines]{c}
  // map PCI ECAM space
  kvmmap(kpgtbl, PCI_ECAM_BASE, PCI_ECAM_BASE, PCI_ECAM_LEN, PTE_R | PTE_W);

  // map PCI PIO
  kvmmap(kpgtbl, PCI_PIO_BASE, PCI_PIO_BASE, PCI_PIO_LEN, PTE_R | PTE_W);

  // map VGA framebuffer (PCI MMIO)
  kvmmap(kpgtbl, VGA_FRAMEBUFFER_BASE, VGA_FRAMEBUFFER_BASE, VGA_FRAMEBUFFER_SIZE, PTE_R | PTE_W);
\end{minted}
These will configure direct mappings for these addresses for kernel read and write access only, denying execution
or unprivileged access. The kernel is now ready to communicate with the emulated PCI VGA device.
\subsection{PCI communication}
\label{chapter2:impl:pci}
\subsubsection{VGA device discovery}
As shown in figure \ref{figure:1}, the PCI configuration space is a table of PCI configuration headers
where the first 32 bits contain the vendor and device IDs. As documented in the QEMU source tree under
/docs/specs/standard-vga.txt \cite{qemusource}, the ID for the emulated VGA card is 0x11111234.

With a simple bruteforce scan through the configuration space, through all 256 PCI buses with 32 devices
per bus, we can compare the first 32 bits of the header with 0x11111234 (see Appendix \ref{appendix:c:3}).

\subsubsection{VGA configuration}
Once the device has been found, we need to write data into the configuration header to achieve the following:
\begin{itemize}
    \item Enable device to respond to port I/O space access, to configure VGA modes and functionality
    \item Enable device to respond to memory space access, for framebuffer access
    \item Configure base address in PCI MMIO address space to expose the framebuffer
\end{itemize}
This can be achieved by the following code snippet, given that `header' is an unsigned 32 bit
pointer to memory.
\begin{minted}[breaklines]{c}
  //-- set bits in the command register
  // Bit 0 - respond to I/O accesses
  // Bit 1 - respond the memory accesses
  // Bit 2 - enable bus mastering (allow card to access other components without CPU intervention)
  header[1] = 0b111;

  __sync_synchronize(); // ensure order of writes aren't modified by optimisation

  // map the framebuffer in our specified base addr
  header[4] = VGA_FRAMEBUFFER_BASE;
\end{minted}

\subsection{VGA graphics driver}
\label{chapter2:impl:driver}
\subsubsection{VGA register access through PCI PIO}
VGA hardware is configured through many 8 bit registers, but the hardware only occupies
a very short port I/O range \cite{osdevvga}. Access is done through indexing, where a register
index is written to a port, then data is written/read for the same or adjacent port. On cases
where data is fed into the same port as the index, an access operation on a specified ``mode setting'' port
will set the port into index mode.

We can see this functionality in the code snippet in listing \ref{listing:2}. The function
handles writing to the VGA registers using ports and register indexes, handling the special cases
mentioned earlier. The read function is also identical, but instead of writing to 
`vga\_pio', the contents of the register index at the given data port offset is returned. Note
that the order of reads and writes are very important here, so we use \mintinline{c}{__sync_synchrnize()}
to emit a memory fence instruction so the compiler and the CPU do not attempt to change the order
of accesses. The \mintinline{c}{__attibute__((unused))} modifier also informs the compiler that
it should not optimise that variable away even though it seems unused, as we need it to perform
a read operation on port 0x3DA to set port 0x3C0 into index mode.

\begin{listing}[H]
    \begin{minted}{c}
static volatile uint8 idx_mode_set __attribute__((unused));
static volatile uint8 *vga_pio = (uint8 *)PCI_PIO_BASE;
static void pwrite(uint32 port, uint32 idx, uint8 val)
{
  if (idx == 0xFF)
  {
    // special case, this port doesn't use indices
    // direct write
    vga_pio[port] = val;
    return;
  }

  // perform a read op on 0x3DA to set 0x3C0 into index mode
  idx_mode_set = vga_pio[0x3DA];
  __sync_synchronize();

  // write the index to the port, and the value to port + 1 if not special case 0x3C0
  vga_pio[port] = idx;
  __sync_synchronize();
  uint write_offset = (port == 0x3C0) ? (uint32)0 : (uint32)1;
  vga_pio[port + write_offset] = val;
  __sync_synchronize();

  // perform a read op on 0x3DA to set 0x3C0 into index mode after
  idx_mode_set = vga_pio[0x3DA];
}
    \end{minted}
    \caption{kernel/pci.c, VGA port write code}
    \label{listing:2}
\end{listing}
\subsubsection{Modesetting}
VGA hardware modes are configured by setting various registers, to configure the timing and resolution
of the output, as well as the colour depth and structure (planar or linear) of the framebuffer.
In non text modes, a palette will also need to be configured for the target colour depth.

For the scope of this project, support for only one mode will be implemented, 320x200 linear 256 colour 
(see Appendix \ref{appendix:c:4} for modesetting value examples).
This mode configures the framebuffer for linear access, with 8 bit (256) colours. This will
also make one pixel equal to one byte, simplifying framebuffer manipulation further in addition to linear access.

Setting the mode can be done by writing the appropriate values to the ports (see kernel/vga.h in the project codebase)
using the port writing logic written in listing \ref{listing:2}.

We will be using the standard VGA 256 colour palette (figure \ref{figure:vgacolor}) for
our implementation. The VGA DAC has a colour subsystem that uses an 18 bit colour
space, 6 bits for each RGB component. We can store this data as a constant in a 32 bit
type array, storing each component in 8 bits (one byte), using only the first 6 bits in each byte.
Writing to the DAC is done in sets of 3, where each component is fed into the DAC data register
sequentially (see listing \ref{listing:palette}).

\begin{figure}[H]
    \centering
    \includegraphics[width=8cm]{1024px-VGA_palette_with_black_borders.png}
    \caption{VGA 256 colour palette}
    \label{figure:vgacolor}
\end{figure}
\begin{listing}[H]
    \begin{minted}[breaklines]{c}
    pwrite(0x3c8, 0xff, 0x00); // set DAC colour register mode to write
    for (int i = 0; i < 256; i++)
    {
      // take 6 bits from relevant part of palette, shift to uint8 type, and write to reg
      // it is 6 bits because the DAC uses an 18 bit colour space, 6 bits per colour component
      pwrite(0x3c9, 0xff, (VGA_256COL_PALETTE[i] & 0xfc0000) >> 18); // set red val
      pwrite(0x3c9, 0xff, (VGA_256COL_PALETTE[i] & 0x00fc00) >> 10); // set green val
      pwrite(0x3c9, 0xff, (VGA_256COL_PALETTE[i] & 0x0000fc) >> 2);  // set blue val
    }
    \end{minted}
    \caption{kernel/vga.c, setting the VGA palette (see listing \ref{listing:2} for pwrite implementation)}
    \label{listing:palette}
\end{listing}

\subsubsection{Boot image}
During system boot, many operating systems display a boot image while it finishes its
boot sequence. Taking inspiration from Linux, the driver will display a boot image, repeated
for every CPU in the system. As each CPU core knows its own CPU number, 
calculating this is trivial, since the width of the screen
and the boot image is known (see Appendix \ref{appendix:c:5}).

\subsection{Grid window manager}
\label{chapter2:impl:winman}

\subsubsection{Device file handling}
As briefly mentioned in Chapter \ref{chapter1:research:winman}, we will be using device files
for programs in userspace to communicate with the window manager. Modifications will need to be
made to file IO handling to register our new device file type (major number) and its handlers
for read/write operations.

Device files are special files that have a major and minor number. For our case,
the major number will indicate that its a `WINDOW' device file with major number 2, with 
the minor number specifying which window the file is referring to. There already
exists another type of device file `CONSOLE' in Xv6 with major number 1; this file serves as an
abstraction to reading and writing from the UART console. Similarly for the
window manager, the new device file will be an abstraction for drawing to a window
and reading the window's input events. Using this design pattern saves us from having
to add more system calls to Xv6 as well.

\begin{figure}[H]
    \centering
    \includegraphics[width=5cm]{fio.drawio.png}
    \caption{Filesystem reads/writes to device files}
\end{figure}

Registering the function handlers to `devsw' is very simple: the addresses for
the read and write handlers are set on
\mintinline{c}{devsw[MAJOR].read} and 
\mintinline{c}{devsw[MAJOR].write} respectively.

Xv6 currently does not pass on the \mintinline{c}{struct file*} pointer.
The function handler signature is \mintinline{c}{(int, uint64, int)}, but we need
details such as the minor number to describe which window and 
the file offset to determine where in the window we start copying data to. We can
simply add the argument for the pointer and modify all the function handlers to 
also expect that argument.

As each window should only be allocated to one program at a time, we can check the
reference count to a device file as it is being opened, and deny access if it already
has an existing reference (another program is using the window).
Since the grid layout also cannot be changed at runtime in the current implementation,
returning the size of the window, the size of the `file' can also be set during opening.
(see listing \ref{listing:winopen})

Note that the function \mintinline{c}{windowman_getsize()} applies some bit tricks
to fit the width and the height into the uint64 type of the file's size. This is done
as the user might find it useful to know the dimensions rather than just the
number of pixels in the window.

\begin{listing}
    \begin{minted}{c}
  if (ip->type == T_DEVICE && ip->major == WINDOW) {
    if (ip->ref > 1) { // only one access per window
      iunlockput(ip);
      end_op();
      return -1;
    }

    ip->size = windowman_getsize();
  }
    \end{minted}
    \caption{Additional checks when opening a window device file}
    \label{listing:winopen}
\end{listing}

\subsubsection{Hooking UART interrupts}
Xv6's only source of input is currently UART. Usually human interaction with
a system is done through keyboard and mouse input, but Xv6 does not support that
and it would be out of scope to implement drivers for those devices. QEMU already
forwards keyboard inputs from the host system as UART inputs, so we will be using
UART as our input.

Currently the console also uses UART for its input as well, as characters are
forwarded to `consoleintr' as they are received in UART (see listing \ref{listing:uartchain}).
We can simply add our window manager handler `windowmanintr' to the chain to also
listen to this input. Note that sometimes we want to `consume' the character 
(not forward it to other listeners), i.e when a window is active the console
should not also receive the input. To achieve this we insert our handler before
the console handler, and check its return value to see if it should call the next
handler in the chain.

\begin{listing}
    \begin{minted}{c}
void
uartintr(void)
{
  // read and process incoming characters.
  while(1){
    int c = uartgetc();
    if(c == -1)
      break;
    if (windowmanintr(c) != 0)
      consoleintr(c);
  }

  // send buffered characters.
  acquire(&uart_tx_lock);
  uartstart();
  release(&uart_tx_lock);
}

    \end{minted}
    \caption{kernel/uart.c, UART interrupt chain}
    \label{listing:uartchain}
\end{listing}

\subsubsection{Window switching}
A mechanism for cycling between the windows to determine which window is active
(if any) is required for multiple reasons. The main use would be to determine
which window's event queue normal UART input is appended to, and to provide
graphical feedback to the user to indicate the active window (see Appendix
\ref{appendix:c:6} for window border drawing code). We also need a
state of no window being active as UART input is also being shared with the console,
and when no window is active the input should be forwarded to the console instead.

As there is no mouse input, hotkeys (UART control characters) 
will be used to cycle between the windows. We can use Control+T to cycle
through a static global in the window manager to keep track of which window
is active (see Appendix \ref{appendix:c:6}). 

\subsubsection{Input event queue}
\label{chapter2:impl:winman:evtq}
Similar to how the console has an input buffer, each window will have an event
queue. Each UART character will count as a seperate input. The data structure
for a window event is as follows:
\begin{minted}[]{c}
struct windowevent {
    enum {EVT_KEY, HEAD, TAIL} type;
    uint64 payload;
};
\end{minted}
The rationale behind having a data structure and a queue is to support other
types of events in the future from other human interface devices, such as a mouse.
Furthermore, the queue will be implemented as a circular queue, to be more space
efficient and reducing the need to shift memory around as events are queued / dequeued.
Since the kernel also does not have a dynamic memory allocator, the size of the event
queue for each window will also be fixed.

\subsubsection{Device read and write functions}
When reading from the window device file, the window manager will instead return
window events from the event queue (see Chapter \ref{chapter2:impl:winman:evtq}) up
to the number requested by the read call. The tail of the event queue will also
be moved as events are dequeued, to make space for new events (see Appendix \ref{appendix:c:7} for code implementation).

When writing to the window device file, user data is copied to the framebuffer sequentially,
from the file offset specified. Since we are copying user data of a variable length,
precautions are needed to defend against potential attacks such as buffer overflows.
In addition to that, checks will need to be implemented to also ensure the write call
is not overflowing their own allocated window, which can potentially draw over
other windows. A single check can address both issues:

\begin{minted}{c}
    // check out of bounds
    const int width = vga_getwidth();
    const int height = vga_getheight();
    if (f->off + n > (width / COLUMNS) * (height / ROWS))
        return -1;
\end{minted}

After the checks are done, data is copied into the window row by row with
\mintinline{c}{either_copyin()}, an existing function that can take a pointer
from a userspace program and copy data from its memory to kernel memory.

\begin{minted}{c}
    while (bytes_left > 0)
    {
        // check bytes left will overflow this row
        if (bytes_left + current_x >= (width / COLUMNS))
        {
            either_copyin(
                framebuffer + (xoffset + current_x) + (yoffset + current_y) * width,
                user_src,
                addr,
                (width / COLUMNS) - current_x);
            bytes_left -= (width / COLUMNS) - current_x;
            current_x = 0;
            current_y++;
            continue;
        }

        // fits in this row
        either_copyin(
            framebuffer + (xoffset + current_x) + (yoffset + current_y) * width,
            user_src,
            addr,
            bytes_left);
        bytes_left = 0;
    }
\end{minted}

\subsection{User mode graphics library}
\label{chapter2:impl:usrlib}

The window manager only offers a very raw and crude way to draw to the window.
Drawing can only be done on contiguous linear blocks of pixels at a variable offset.
To address this, we can write a graphics library in usermode to provide some 
abstractions such as text rendering, drawing rectangles, rectangular sprites, etc.

Creating an abstraction layer that is designed around the functionality of a window
will also be useful for future purposes if the underlying implementation details were to
change. 

\subsubsection{Automatic window allocation and destruction}
Since a window can only be accessed at one program at a time, the process of finding
a free window can be tedious and will most likely be repeated for all user programs.

We can abstract this logic into a \mintinline{c}{window_create()} function that
returns a handle to the window. The function will find all device files in the
root directory that starts with `window'. If one opens successfully, a data structure
defined internally in the library is created containing the file descriptor
and the dimensions, and returned to the user in the form of a opaque pointer
(pointer to forward declared struct).
We don't want users of the library to directly access this structure in case
the implementation changes, therefore we expose it as an opqaue pointer
for internal use only.

A complementary \mintinline{c}{window_destroy()} is also written that clears
the window, closes the underlying file descriptor, and frees the handle (struct pointer).

This structure takes inspiration from the `CreateWindow()' and `DestroyWindow()' 
Windows API pattern \cite{windowsapi:window}.

\subsubsection{Event polling}
Our window manager makes event polling very simple: just perform a read syscall
with the window device file descriptor with the desired maximum events to read.
However since the provided window handle is an opaque type, the library will provide
a \mintinline{c}{window_pollevent()} with a function signature identical to the read syscall,
with the exception of the first argument taking the window handle instead of an integer 
descriptor. The function will forward the arguments to the read systemcall and return
the number of events read. 

Using this method not only conforms to the design pattern of the library, but
has the added benefit of clearer self-documenting code instead of an ambiguous
read call to the window descriptor.

\subsubsection{Text rendering}
The 8x16 font is stored in kernel/font.h where each glyph is encoded in rows where 
each row is a byte and a bit value of 1 indicates foreground and 0 indicates background.

\begin{listing}[h]
    \begin{minted}{c}
void window_drawchar(window_handle win, uint x, uint y, char c, uint8 color)
{
    int mask[8] = {128, 64, 32, 16, 8, 4, 2, 1};
    uint8 *glyph = VGA_8x16_FONT + (int)c * 16;

    for (int j = 0; j < 16; j++)
    {
        for (int i = 0; i < 8; i++)
        {
            if (x + i >= win->width || y + j >= win->height)
                continue;
            
            if (glyph[j] & mask[i])
            {
                seek(win->fd, SEEK_SET, (y + j) * win->width + x + i);
                write(win->fd, &color, 1);
            }
        }
    }

    // reset cursor
    seek(win->fd, SEEK_SET, 0);
}
    \end{minted}
    \caption{user/uwindow.c, character rendering logic}
    \label{listing:charrender}
\end{listing}

As shown in listing \ref{listing:charrender}, character rendering is actually 
quite trivial. Given a character `c', we get the offset in the ASCII font, and
then decode the glyphs. The mask is applied for every pixel as each 8 bits
is actually an encoded row. The foreground pixels are then written to the screen
with the specified colour and coordinate offset.

Note that this implementation is potentially slow since worst case it may result in 
8 * 16 * 2 system calls.

\subsubsection{Shape rendering}
The ability to draw shape primitives is essential to creating a basic graphical interface.
We will implement functionality to draw rectangles, circles and lines.

Drawing rectangles is simple: we simply use \mintinline{c}{seek()} to draw the rectangle
row by row, ensuring we don't overflow the current row.

\begin{minted}{c}
    uint8 *rowbuf = malloc(sizeof(uint8) * width);
    for (int j = 0; j < height; ++j)
    {
        // copy color into rowbuf and write
        memset(rowbuf, color, width);
        seek(win->fd, SEEK_SET, ((y + j) * win->width) + x);
        write(win->fd, rowbuf, width);
    }
\end{minted}

Drawing circles is also quite simple, although not as performant. Each pixel
in the square bounding box of the circle is checked against the distance
from the specified centre coodinate, and drawn if within the circle radius. 

\begin{minted}{c}
    uint rsquared = r * r;
    for (uint j = y - r; j < y + r && j < win->height; ++j)
        for (uint i = x - r; i < x + r && i < win->width; ++i)
        {
            uint dx = x - i;
            uint dy = y - j;
            uint distsquared = dx * dx + dy * dy;

            if (distsquared <= rsquared)
            {
                seek(win->fd, SEEK_SET, j * win->width + i);
                write(win->fd, &color, 1);
            }
        }
\end{minted}

Drawing lines is a slightly more complicated process, involving interpolating
pixels between the two specified points. This requires floating point operations
as the interval between steps is fractional. It is also not very performant as
we draw each pixel individually.

\begin{minted}{c}
    int steps = (abs(y1 - y2) + abs(x1 - x2));
    float interval = (float)1 / steps;

    float t = 0;
    for (int i = 0; i < steps; i++, t += interval)
    {
        // this is really slow, kernel doesn't support masking / blitting
        // yet so if we don't want to overwrite a rectangular region this is
        // the only way for now
        seek(win->fd, SEEK_SET, interpolate(y1, y2, t) * win->width + interpolate(x1, x2, t));
        write(win->fd, &color, 1);
    }
\end{minted}

\subsubsection{Sprite rendering}
Rendering rectangular sprites is identical to rendering rectangles, but instead
of using a solid colour we use the sprite data for the colours.

\begin{minted}{c}
    uint8 *rowbuf = malloc(sizeof(uint8) * width);
    for (int j = 0; j < height; ++j)
    {
        // copy sprite bytes into rowbuf
        memmove(rowbuf, data + (j * w), width);

        // write to screen
        seek(win->fd, SEEK_SET, ((y + j) * win->width) + x);
        write(win->fd, rowbuf, width);
    }
\end{minted}

\section{Scheduler modifications}
Processes that are drawing graphics should be prioritised as to appear more
responsive for the user experience. The current scheduler is a round robin 
scheduler that may make graphical applications appear less responsive if there are a
lot of background processes running.

To address this, a simple priority scheduler will be implemented. Each process will
have a priority flag, and the new scheduler algorithm will run priority processes
first also in a round robin fashion, otherwise it will run the first available
process also with round robin.

Small modifications to kernel/proc.h will be required to support priority processes.
Specifically, the priority needs to be added to the processes data structure and a
process state enum of `RESERVED' will also be needed.

\noindent The new algorithm will be as follows (see Appendix \ref{appendix:c:8} for code):
\begin{enumerate}
    \item Is the current process a priority process? Go to 6 if yes.
    \item Is the current process a runnable process? Save the process as a reserve if yes.
    \item Go to the next process
    \item Have we reached our starting point since last run? Set the reserve process (if exists) and go to 6.
    \item Go back to 1.
    \item Run process.
    \item Go to the next process.
    \item Go back to 1.
\end{enumerate}

Finally, a new system call will need to be added to set the running process' 
priority. This is very simple to do, as the new syscall will only need to set
the \mintinline{c}{p->priority} value to the first argument of the call. See
Appendix \ref{appendix:c:8} for the detailed modifications.

\section{Version control}
Illustrated in the commit graph (figure \ref{figure:commitgraph}), proper version control
with a defined workflow was used. `Feature' branches contain work on mostly isolated
features, the `develop' branch contains work from multiple feature branches, and 
finally the `master' branch has merges from `develop' when the build is stable.

Although this workflow is more suited towards projects where features can be developed
concurrently with potentially multiple team members, it provides a nice structure
to the commit history. Commits that contribute to a feature can be easily grouped,
and there is a dedicated branch for stable builds.

\begin{figure}[h]
    \centering
    \includegraphics[width=12cm]{Screenshot from 2022-04-27 12-47-49.png}
    \caption{Git commit graph}
    \label{figure:commitgraph}
\end{figure}
